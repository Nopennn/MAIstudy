\documentclass{article}
\usepackage[utf8]{inputenc}
\usepackage[russian]{babel}
\usepackage{indentfirst}
\usepackage{setspace}
\usepackage{amsmath}

\begin{document}
\setcounter{page}{351} 
\setcounter{section}{13} 
\setcounter{subsection}{3} 
\linespread{1.01}
\begin{large}
\parindent=0cm Здесь были использованы разложения по формуле Тейлора:

\parindent=0.6cm косинуса
$$\cos x = 1 - 	\frac{x^2}{2}+O(x^4), x	\rightarrow0,$$

синуса
$$\sin x = x - 	\frac{x^3}{6}+O(x^5), x	\rightarrow0,$$

бинома
$$(1+u)^\alpha = 1+\alpha u+O(u^2)$$

\parindent=0cm при $\alpha = -1$ и $u = -\frac{x^2}{6} + O(x^4), x	\rightarrow0.$

\subsection*{\textbf{13.4. Вычисление пределов\\с помощью формулы Тейлора\\(метод выделения главной части)}}

Формула Тейлора даёт простое и весьма общее правило для выделения главной части функции. В результате этого метод вычисления пределов функций с помощью выделения главной части приобретает законченный алгоритмический характер.

\parindent=0.6cm Рассмотрим сначала$\ $ случай$\ $ неопределенности$\ $ вида $\ \frac{0}{0}$.

\parindent=0cm Пусть требуется найти предел 	$\ \lim\limits_{x\to x_0}\frac{f(x)}{g(x)}$, где $\ \lim\limits_{x\to x_0}\ f(x)=\\=\lim\limits_{x\to x_0}\ g(x)=0$. В этом случае рекомендуется разложить по формуле Тейлора функции $f$ и $g$ в окресности точки $x_0$ (если,$\ \ $ конечно,$\ $ это$\ $ возможно),$\ $ ограничившись в этом разложении лишь первыми не равными нулю членами, т.е. взять разложения в виде
$$f(x)=a(x-x_0)^n+o((x-x_0)^n), a\not=0,$$
$$g(x)=b(x-x_0)^m+o((x-x_0)^m), b\not=0,$$
тогда
$$\lim\limits_{x\to x_0}\frac{f(x)}{g(x)} = \lim\limits_{x\to x_0}\frac{a(x-x_0)^n+o((x-x_0)^n)}{b(x-x_0)^m+o((x-x_0)^m)} =$$ 

$$ = \frac{a}{b}\lim\limits_{x\to x_0}(x-x_0)^{n-m}= f =\begin{cases} 0, n>m \\  \frac{a}{b}, n = m \\ \infty, n<m \end{cases} $$

$$\\ \\ \\ \rule{65pt}{0.01cm}$$

\newpage

\setlength{\parindent}{0.7cm} Часто бывает удобно$\ $ для $\ $разложеия$\ $ функций $f$ и $g$ по формуле Тейлора$\ $ использовать $\ $готовый$\ $ набор разложений элементарных$\ $ функций,$\ $ полученный в $\ $п. 13.3.$\ $ Для этого следует в случае$\ x_0 \not= 0\ $ предварительно$\ $ выполнить замену переменного $\ t=x-x_0$;$\ $ тогда $\ x\to x_0\ $ будет соответствовать $\ t\to 0$. Случай $\ x\to \infty \ $ заменой переменного $\ x = 1/t \ $ сводится к случаю $\ t\to 0$. 

Если имеется $\ $неопределенность вида $\ \frac{\infty}{\infty}\ $, $\ $т.е. $\ $требуется найти $\ \lim\limits_{x\to x_0}\frac{f(x)}{g(x)}$ , где $\ \lim\limits_{x\to x_0}\ f(x)=\lim\limits_{x\to x_0}\ g(x)=\infty$, то ее легко привести $\ $к$\ $ рассмотренному $\ $случаю $\ \ \frac{0}{0}$ $\ \ $преобразованием $\ \frac{f(x)}{g(x)} = \frac{1/g(x)}{1/f(x)}$.

Подобно $\ $вычислению пределов$\ $ с помощью $\ $правила$\ $ Лопиталя, при применении$\ $ метода выделения $\ $главной части к раскрытию неопределенностей вида $\ 0\ \cdot \  \infty\ $ и $\ \infty\ -\ \infty\ $ их следу\-ет преобразовать к неопределенности вида $\ \frac{0}{0}$. Наконец, для раскрытия неопределенностей вида $\ 0^0, \infty^0$ и $\ 1^\infty$ указанным методом необходимо предварительно прологарифмировать рассматриваемые$\ $ функции. $\ $Посмотрим $\ $на $\ $примерах, как применяется формула Тейлора к вычислению пределов функций. Пусть требуется найти
$$\lim\limits_{x\to x_0}\frac{e^x-e^{-x}-2x}{x-\sin x}.$$
Заметив, что(см. п. 13.3)
$$e^x = 1+x+x^2/2+x^3/3+o(x^3),$$
$$e^{-x} = 1-x+x^2/2-x^3/3+o(x^3),$$
$$\sin x = x-x^3/3!+o(x^3),$$
получим
$$\lim\limits_{x\to 0}\frac{e^x-e^{-x}-2x}{x-\sin x} =$$
$$= \lim\limits_{x\to 0}\frac{x^3/3+o(x^3)}{x^3/6+o(x^3)}= \lim\limits_{x\to 0}\frac{x^3/3}{x^3/6}=2.$$


$$\\ \\ \\ \rule{65pt}{0.01cm}$$
\end{large}
\end{document}

